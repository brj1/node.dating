% !TEX TS-program = xelatex
\documentclass[12pt]{article}

\usepackage[margin=2cm]{geometry}
\usepackage{colortbl}
\usepackage{comment}
\usepackage{caption}
\usepackage{subcaption}
\usepackage{mathptmx}
\usepackage{nicefrac}

\usepackage{times}
\usepackage{lineno}
\usepackage[round]{natbib}
\makeatletter
\renewcommand{\@biblabel}[1]{#1.}
\makeatother

\usepackage{graphicx}
\usepackage{amsmath}
\usepackage{xcolor}

\usepackage{url}
\urlstyle{same}

\usepackage[small,compact]{titlesec}

%\titleformat{\section} {\vspace{24pt}\bf\sffamily\MakeUppercase}{\thesection} {0pt} {}
\titleformat{\section} {\vspace{12pt}\bf\large}{\thesection}{0pt}{}
\titleformat{\subsection} {\vspace{6pt}\bf}{\thesubsection} {0pt} {\vspace{-2pt}}
\titleformat{\subsubsection} [runin] {\bf}{\thesubsubsection} {12pt} {}

\newcommand{\code}[1]{{\tt #1}}
 
\begin{document}

\title{node.dating: dating ancestor nodes in phylogenetic trees}

\author{Bradley R. Jones \\ Simon Fraser University, Burnaby, Canada \\ BC Centre for Excellence in HIV/AIDS, Vancouver, Canada}
%%% 1 Simon Fraser University
%%% 2 BC Centre for Excellence in HIV/AIDS
\baselineskip 22pt
\pagewiselinenumbers

\date{}
\maketitle

\section * {Abstract}

Keywords: 
Phylogenetics, Ancestor Dating, Linear Regression \\



\underline{}
\section*{Introduction} \label{sec:intro}

%%%%%%%%%%%%%%%%%%%%%%%% METHODS %%%%%%%%%%%%%%%%%%%%%%%%%%

\section*{Methods} \label{sec:methods}

\subsection*{Algorithm} \label{sec:alg}
A simple linear regression is used to estimate the mutation rate assuming a strict molecular clock.

To estimate the dates of the internal nodes, I follow an approach given by \cite{Felsenstein81}. A maximum likelihood method is applied locally to date each internal node. Then using these estimates, the algorithm iterates until an approximate maximum likelihood solution is obtained.

\subsection*{Implementation} \label{sec:impl}
\code{node.dating} was implemented in R to be part of the APE package \citep{APE}.

\subsection*{Simulation} \label{sec:sim}
To verify the accuracy of \code{node.dating}, I applied it to simulated data. I simulated 50 phylogenetic trees using a birth death model with the the R package \emph{TreeSim} \citep{TreeSim}. I then applied a strict molecular clock to trees wth the R package \emph{NELSI} \citep{NELSI}. I used these trees to generate simulated HIV sequences with \emph{INDELible} 1.03 \citep{Indelible09}. Finally I reconstructed phylogentic trees from the sequences using \emph{RAxML} 8.2.4 \citep{Raxml14}. This process is engineered to replicate phylogenetic trees derived from real data.

\section*{Results} \label{sec:results}
\subsection*{Comparison with TempEst} \label{sec:tempest}
In order to test the viability of \code{node.dating}, I compared it with \emph{TempEst}'s \citep{Tempest} node dating.



%%%%%%%%%%%%%%%%%%%%%%%%%%%%%%%%%%%%%%
\section*{Discussion} \label{sec:discuss}

Future work includes allowing node dates to be seeded/fixed and adding a relaxed molecular clock.

\section*{Acknowledgments} \label{sec:ackn}
%This work was supported by the Canadian Institutes of Health Research (CIHR Team Grant: HIV Cure Research --- The Canadian HIV Cure Research Enterprise; CanCure) and by a CIHR operating grant to Art Poon (HOP-111406).

\clearpage

\bibliographystyle{plainnat}
\bibliography{node.dating}


\clearpage

%%%%%%%%%%%%%%%  FIGURES  %%%%%%%%%%%%%%%

%%%%%%%%%%%%%%%  TABLES  %%%%%%%%%%%%%%%

\end{document}
